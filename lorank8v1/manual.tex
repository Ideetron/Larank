\documentclass[12pt]{article}
\usepackage[a4paper,margin=20mm]{geometry}
\usepackage[parfill]{parskip}
\usepackage{graphicx} 
\usepackage{sidecap}
\usepackage{subfig}
\usepackage{color}
\usepackage{url}
\usepackage{mathabx}
\usepackage{relsize}
\usepackage{tcolorbox}
\usepackage{adjustbox}
%\usepackage{hyperref}
%\usepackage[usenames,dvipsnames]{pstricks}
%\usepackage{epsfig}
%\usepackage{pst-all}
%\usepackage{pst-pdf}
%\usepackage{pst-grad} % For gradients
%\usepackage{pst-plot} % For axes
\usepackage{setspace}
\usepackage{emp}
\usepackage{type1cm}
\usepackage{eso-pic}
\usepackage{multirow}
\usepackage{hhline}
\usepackage{colortbl}
\usepackage{tabularx}
\usepackage{arydshln}
\usepackage{xcolor}
\usepackage{lscape}
\usepackage{enumitem}

\newcommand{\ina}{\mbox{\hspace{2mm}}}
\newcommand{\inb}{\mbox{\hspace{4mm}}}
\newcommand{\inc}{\mbox{\hspace{6mm}}}
\newcommand{\ind}{\mbox{\hspace{8mm}}}
\newcommand{\pha}{\phantom\{}
\newcommand{\phb}{\phantom\{\phantom\{}
\newcommand{\phc}{\phantom\{\phantom\{\phantom\{}
\newcommand{\phd}{\phantom\{\phantom\{\phantom\{\phantom\{}
\newcommand{\nl}{\newline}

\title{Manual Lorank 8.}
\author{Ideetron}
\date{\today}

\input{../Commands-Common}

\begin{document}
\maketitle

\begin{center}
\includegraphics[width=70mm]{pic/redsmall.jpg}\\
\larger[1] Document Version 1.0.7 \\
\smaller[1] See \url{https://github.com/Ideetron/Lorank} \\ for the latest version of this manual.
\end{center}


\newpage

% ==========================================================================================================================
\section{Introduction}
Welcome to the Lorank 8. This manual presents you with all the necessary 
information to run your Lorank 8 in a safe way. Please read this manual carefully 
before starting the Lorank 8 for the first time. 

Lora technology is a very young technology and as such constantly changing. 
The Lorank 8 implements the latest hardware and software as well 
as the most recent publicly available protocol specifications such
as LoRaWAN\_1R0. In order to be kept informed of future updates 
you can subscribe to the Github of Ideetron: \url{https://github.com/Ideetron/Lorank}.  

\section{Quick start}
To quickly set up your gateway perform the steps below. 
For this quick start to work, the following must be fulfilled:
\begin{itemize}
\item On your local network a DHCP server with DNS capabilities must be operational.
\item Your PC and gateway are on the same subnet.
\item Your router's firewall does {\bf not} block outgoing data traffic on higher port numbers.
\item Make use of a recent browser (preferably {\bf not} Internet Explorer).
\end{itemize}

Then we can test if the gateway works and receives and forwards packets.
If your router does not have DNS capabilities, the IP number must be used, please log
in to you router to obtain the number that was issued to the gateway.

\begin{itemize}
\item Mount the antenna ({\bf never} operate the gateway without an antenna!)
\item Connect the gateway to your local network using an ethernet cable.
\item Connect the gateway to the supplied power source, and give it a few minutes to boot.
\item On a pc, surf to \url{http://lorank8/} or \url{http://lorank8.local/} or \url{http://[ip-number]/}
\end{itemize}

Now you should see a simple webpage with a picture of the Lorank device. 
If this is all correct, we can see if the data gets outside, to that end:

\begin{itemize}
\item Switch on any LoraWAN capable Mote [keep at least (!!) 2 meters distance between Mote and Gateway].
\item Surf to \url{http://iot.semtech.com/gateways/} to see if the gateway is up, the first two bytes are 1D-EE.
\item Surf to \url{http://iot.semtech.com/motes/} to see the Mote data.  
\end{itemize}
If you can find your gateway in the list, and see the Mote's data, you know everything works
as required. If you see the gateway but not your Mote, the latter may not be functioning.
If any of these steps should not succeed a more in careful startup procedure is needed, please see below.
If you do not have a Mote, the gateway may not show up in the gateway list on \url{iot.semtech.com},
in that case see if it is visible on the map on their website.

Subsequently you can test if these packets are visible on The Things Network:
\begin{itemize}
\item Surf to \url{http://www.ttnstatus.org/} to see if the gateway is up.
\item Surf to \url{http://thethingsnetwork.org/api/v0/gateways/} for additional data.
\item Surf to \url{http://thethingsnetwork.org/api/v0/nodes/} to see the Mote data.
\end{itemize}
The Things Network is not yet fully operational and hiccups have been reported. 
In case that seems to be at hand, but the former step works, try again after a few days.

\section{Connection to the network}
The gateway should first be connected to a local network and subsequently connected to the power adapter.
Do {\bf not} connect the gateway directly to outside Internet. First, that network is very busy, and,
more importantly, for the preconfigured gateway, it is relatively easy for outsiders to break in.
Once that happens, a complete reflash of the memory is needed. The router takes care of the primary
protection of your gateway. 

The blue leds should light up. If this does not happen, there is something wrong with the
5V Stabilised Power Supply. Please use the one that was delivered with the unit. 

After a few minutes, the gateway starts asking for an IP number on the network using its DHCP client. 
It is expected that you have a DHCP server running on your network. 
Routers from network providers usually are configured that way. Depending on the services provided
by that router you can find the gateway on its IP number or try to address it by its hostname.

Try in a recent browser (preferably not Internet Explorer!) \url{http://lorank8/} or \url{http://lorank8.local/}
You should see a webpage with a picture of the Lorank device. This indicates that the gateway is running.

If that does not work, the router is probably not resolving  the name because no local dns is 
running. In that case you need to obtain the IP number. This can only be done by logging into 
the router itself, and look upon the list of connected devices. Please refer to the manual of 
your router how this can be done. The number is usually something like 192.168.x.y or 
10.x.y.z where x,y,z are numbers between 0 and 255. Then, in the browser try  \url{http://192.168.x.y/}

If more than 10 minutes have past, and still no IP number is visible in the routers connection
list, take out the power cable (from the gateway, not the mains) and reconnect. Occasionally it
can happen dat DHCP server and client misunderstand each other. Do not try multiple reconnections 
of the power quickly in a row, this can damage the internal file system.

Apart from browsing to the webserver of the gateway, one can also login to the gateway using ssh.

\section{Basic configuration}
Although the gateway runs out of the box, further configuration may be required. 
Besides that, all gateways come with the same predefined root password, which is a potential 
security issue, and thus this must be changed. Furthermore, you may want to change the routers 
to which all data is send.

\subsection{Default routers}
The current version of the software allows for maximally four routers to be configured as 
targets for the data. Note that, in order for the service to actually start, {\bf all} 
these routers  must exist, although they do no have to accept the data. This is a shortcoming 
of all current packet forwarders, which we expect to be lifted in the next version. 
Per default we have configured the following targets in the global configuration file:
\begin{itemize}
\item {\bf Semtech}: This company hosts a router that all gateways may utilise for demonstration purposes,
see \url{www.semtech.com} for more information. 
\item {\bf The Things Network}: This is a crowd sourced IOT network, open for anyone to use, see 
\url{www.thethingsnetwork.org} for more information.
\item {\bf Loraley}: The is an alternative distributed open source IOT data network, about to be 
launched  in the coming months, see \url{www.loraley.org} for more information.
\end{itemize}
Please feel free to use one or more of the preinstalled routers or direct the data to your
own private network. Note that, in the future, these url's may change and thus this may require
further attention. Consult the websites of the organisations you want to make
use of first.

\subsection{The Gateway ID}
If the gateway is part a larger network (when not operated privately) it must identify itself 
using an eight byte {\bf worldwide} unique identifier. The device comes with such a number 
preconfigured, starting with `1DEE'. This number is stored in the files \url{gatewayID}
and in \url{local_conf.json} located in the directory that also contain the executables.

It is possible to change this identifier manually if
needed, just by editing these files (see below how). However, if the device is to be operated on
other networks such as The Things Network, we
advise not to do so, or to make sure to choose an identifier that is guaranteed to be globally
unique. In practice the only reliable way to ensure this is to construct it based on machine generated
random numbers. In any case do {\bf not} choose something based on any name, simple sequences 12345
or something alike. At this moment there are not a lot of gateway's active around the world, 
so it may not lead to problems immediately, but this will certainly change in the near future.
Dataloss or impossibility to reach your nodes may be the consequence.


\subsection{Configuration parameters}
Depending on the packet forwarder you run (per default, the poly forwarder is running),
you can set several parameters. Below find an overview, together  with their meaning and
default setting. There are more, but these are beyond the scope of this manual. The
relevant ones are:

\smaller[2]
\begin{verbatim}
/* Devices */
"gps"         : true      /* Indicate if you want to include (static) gps coordinates in the stream. */
"monitor"     : false     /* If you have monitor software running, activate connection. */
"upstream"    : true      /* Set to true if you want to be able to receive data from the nodes. */
"downstream"  : true      /* Set to true if you want to be able to send data to the nodes. */
"ghoststream" : false     /* Set to true if you have a fake packet generator running */
"radiostream" : true      /* Set to true if you have a concentrator connected */
"statusstream": true      /* Set to true if you want to include status updates */

/* Set a globally unique ID */
"gateway_ID": "1DEE000000000000"  /* This number is globally unique for each device. */

/* node server for basic packet server, used by basic packet server or when no other are available */
"server_address": "iot.semtech.com"              /* domain name or ip server of netwerk server */
"serv_port_up"  : 1680                           /* port for upstream data */
"serv_port_down": 1680                           /* port for downstream data */ 
       
/* node servers for poly packet server (max 4 enabled), read by poly packet server only */
"servers":
[ { "server_address": "croft.thingsnetwork.org"  /* domain name or ip server of first netwerk server */
    "serv_port_up"  : 1700                       /* port for upstream data */
    "serv_port_down": 1701                       /* port for downstream data */ 
    "serv_enabled"  : true }                     /* enable this server */
  { "server_address": "amsterdam.loraley.org"    /* domain name or ip server of second netwerk server */
    "serv_port_up"  : 1680                       /* port for upstream data */
    "serv_port_down": 1681                       /* port for downstream data */ 
    "serv_enabled"  : true } ]                   /* enable this server */
                
/* GPS configuration */
"ref_latitude" : 0   /* Enter the latitude of the location your gateway is mounted */
"ref_longitude": 0   /* Enter the longitude of the location your gateway is mounted */
"ref_altitude" : 0   /* Enter the altitude of the location your gateway is mounted */

/* Ghost configuration */
"ghost_address": "127.0.0.1"     /* domain name or ip server of the ghost data server */
"ghost_port"   : 1918            /* connection port for fake packets */ 

/* Monitor configuration */      
"monitor_address": "127.0.0.1"   /* domain name or ip server of the ghost data server */
"monitor_port"   : 2008          /* connection port for machine controle */ 

/* Informal data for status updates. */
"platform"      : "*"                     /* Platform definition, put * for internal, max 24 chars. */
"contact_email" : "operator@gateway.tst"  /* Email of gateway operator, max 40 chars */
"description"   : "Update me"             /* Public description of this device, max 64 chars */


\end{verbatim} 
\larger[2]
 
These options can be set in different manners, see the section below for the possibilities. 

\subsection{Configuration with builtin the webserver}
Planned for future versions, not available this release, please configure using SSH. 

\subsection{Configuration with SSH over the command line.}
Per default a SSH service runs on the gateway, making is possible to login and modify the 
settings and software from the command line. The standard login credentials are:
\begin{verbatim}
  account: root
 password: LorankAdmin
\end{verbatim}
You can login with ssh with the command ``{\bf ssh root@lorank8}'' or ``{\bf ssh root@[IP number]}''.
The first action to take is to change the root password into something personal:
\begin{verbatim}
 lorank8 # passwd 
  (current) password: LorankAdmin
  Enter new password: ***********  [Type something sensible]
 Retype new password: ***********  [Retype your password] 
\end{verbatim}
Please make {\bf sure} you remember or write down this password. If forgotten, the gateway
cannot be enterend again, Ideetron does {\bf not} have any means of recovery. 
The only solution in that situation would be a complete 
reinstall of the Beagle Bone and subsequent installation of Lorank gateway 
software. This is a process requiring expert skills. See \url{http://beagleboard.org} 
and \url{https://github.com/Ideetron} for more information.

The Lorank software is located in the root directory for convenience. Usually there are
three directories. One build directory, something like `lorank-v1.0.4', a system directory
called `Lorank', and a platform dependent working directory that contains the executables and
setting files for your platform. This is called `Lorank8v1' or something alike. The work
directory contains the executables needed to run the gateway as well as its configuration files.
These files are called \url{global_conf.json} and \url{local_conf.json}. Entries in the
latter file supersede the ones in the former. Although the files end in `.json', they are
not JSON files in a strikt sense. However, if you do edit these files, make sure you
adhere to the format used, for otherwise the forwarder will not be able to run.
Standard editing tools like \url{nano} and \url{vi} are available on the platform,
for example
\begin{verbatim}
  nano global_conf.json
\end{verbatim}
Make your changes using the arrow keys (mouse does not work) and save with \^{}O
(control-O), exit with \^{}X (control-X).

Per default, the poly forwarder is running in the background. Output is send to the 
system logger. If you want to experiment, it is better to test the forwarder in the
foreground. In that case, the background must be stopped first. The configuration files are
only read at the start of the forwarder.

The background forwarder can managed using the following commands:
\begin{verbatim}
  systemctl start lorank.service
  systemctl stop lorank.service
  systemctl restart lorank.service
  systemctl status lorank.service
\end{verbatim}

To enable or disable the forwarder startup at boottime use the commands:
\begin{verbatim}
  systemctl enable lorank.service
  systemctl disable lorank.service
\end{verbatim}
 
 
 
To test a forwarder in the foreground, for example after you have changed
some configuration parameters, simply stop the background forwarder en 
start the new forwarder in the foreground like this:
\begin{verbatim}
  systemctl stop lorank.service
  ./poly_pkt_fwd
\end{verbatim}
The output will be put directly on screen, so you can see if the new configuration
performs as expected. If so, stop it with \^{}C (control-C) and reactivate the 
background forwarder.
\begin{verbatim}
  ^C
  systemctl start lorank.service
\end{verbatim}

All logging of the service is written to \url{/var/log/syslog}, so
to see what is happening under the hood, check, for example the last 
200 lines in this file with:
\begin{verbatim}
  tail -n 200 /var/log/syslog
\end{verbatim}
 

\subsection{Via SSH using PuTTY.}
No version of Windows OS contain the SSH utility out of the box. So the 
best way to log into the gateway is making use of PuTTY, which can be 
found here: \url{http://www.putty.org/}.   


\section{Hardware}
The gateway is build upon the Concentrator Board of IMST iC880A and a
recent Beagle Bone Board. The latter is completely open source hard- and software, 
see \url{http://beagleboard.org} for more information. In 
between a connection board has been placed. The Beagle Bord's usb is accessible
from the outside and may not be loaded with more that 500 mA.


\section{Software}

\subsection{Location}
All software is opensource and can be found on the Github of Ideetron:
\url{https://github.com/Ideetron}. There are three main repositories
that are relevant
\begin{verbatim}
  https://github.com/Ideetron/Lorank
  https://github.com/Ideetron/packet_forwarder
  https://github.com/Ideetron/lora_gateway
\end{verbatim}
Beware however, that at the time of reading these may have different names, be moved
or there may be added repositories. Please read the latest README's in the 
repositories. 

\subsection{Installation}
The gateway come preinstalled. However it can be necessary to reinstall or upgrade
the software. Complete reinstallation is a delicate procedure that requirers 
experience with embedded systems. Although the command sequence is easy, differences
between platforms may cause serious trouble. Therefore we strongly discourage this
procedure in general, and is given here only for completeness. The least you should
do before exercising this is read the scripts called. If there is anything in there
you do not understand, please do not move forward. 
\begin{verbatim}
  ./Lorank/lorank8v1/wipe   
  git clone https://github.com/Ideetron/Lorank.git; 
  ./Lorank/install lorank8v1 1.2.3
\end{verbatim}
Note that \url{wipe} {\bf completely removes all Lorank software and settings}.
Replace \url{1.2.3} with the software version you want to install. These can
be found as releases on Github.

\subsection{Upgrade}
All executables are installed and build form source using git, so this can
be used to upgrade as well. It is also possible, and a maybe safer if you are
not comfortable with git to install a new version of the 
software. This does not remove your old version, so if something goes wrong,
you can revert. Still, some level of expertise is needed here, should something 
go wrong. Basically the steps are (in the directory \url{/root})
\begin{verbatim}
  cp -r lorank8v1 lorank8v1.backup
  ./Lorank/lorank8v1/upgrade 1.2.3  
\end{verbatim}
If no release version is provided the most recent code is used. Usually this is not
a good idea, since this code is under heavy development. Please use the latest
{\bf release version} as can be found on Github. The new version is installed in a
separate directory and the executables are copied to the working directory.
Since the former version stays on your system, you can revert by manually
coping those back to the working directory. An other, simpler solution is to
make a backup of the working directory, as is done above.
When performing the upgrade, the \url{global_conf.json} is replaced, the old one 
saved as \url{global_conf_old.json} in the same directory. 
The \url{local_conf.json} is not touched because it contains your Gateway 
Unique Identifier. Usually some merging of the old en new \url{global_conf} file
is needed. This must be performed by hand. Please first test the new software 
and its configuration before putting it into production.
Should you be on an older version (prior to 0.1.5) that does not have the 
upgrade script yet, please update that first using git:
\begin{verbatim}
  cd Lorank
  git checkout master
  git pull
  cd ..  
\end{verbatim}
and then proceed as above. 

\section{Operation}

\subsection{Antenna}
{\bf\textcolor{red}{Warning}}: Never operate the gateway without a properly mounted antenna.
If not mounted the energy meant for transmission reflects back into the
device and immediately destroy the input stages. The gateway will be 
damaged beyond repair.

\subsection{Always on}
The Lorank is designed for 24/7 operation, provided a few precautions are
taken. Do not place the device into direct sunlight (be aware of the 
change of sunlight during the day), near heaters or in conditions of 
condensing moisture. The  Lorank is best mounted on a high location 
for optimal reception, away from metal objects. Mount the device such 
that the antenna points upwards.  The casing should at least be free 
on two of the sides so that the heat produced is sufficiently drained.
Other transmitters such as Lora Nodes, WiFi stations and cellphones 
should at least be at two meters distance from the Lorank to prevent
damage to its highly sensitive input amplifiers. 

Occasionally the server that accepts your data may stop doing so, even
while the packets are send from the gateway without interruption. 
This can happen for instance when a crash happens serverside or a firewall
that is in the pathway is reinitialised. Unfortunately, the gateway is 
not informed of such an event, and the only option for now is to restart
your gateway. See above how this can be done. We expected this
problem to disappear as the software in the whole chain gets more
mature over time.  

\subsection{Shutting down}
The Lorank does not have a power switch, so it seems natural to simply
`pull the plug' it when it has to be switched off. Although this is 
possible, and will not directly do harm it is better to login into 
the gateway via SSH, as described above and bring the gateway down 
in an orderly fashion. This can be done with the command:
\begin{verbatim}
  shutdown -h now
\end{verbatim}
The gateway will start a shutdown procedure and after approximately 
30 seconds all leds should be off. Restarting can be done by
simply disconnecting and reconnecting the power cable.


\section{Specifications}

\begin{verbatim}
Hardware:
- Frequency band        : 868 MHz
- Sensitivity           : -138 dBm
- Maximum power         : 27 dBm (500mW)
- LoRa demodulators     : 49
- Simultaneous channels : 8
- Max connected nodes   : ~60 thousand (*)
- Processor             : 1GHz, ARM Cortex A8
- OS                    : Debian / Angstrom Linux
- Wifi                  : Optional (via usb)
- Current               : 1A
- Max Current USB       : 500mA [USB is internal]
- Power Adapter         : 5Volt= , 2Amp 

(*) This is a theoretical maximum, under the assumption that nodes
    only send once per hour. Due to collisions, resend packets, packet
    loss etc, the number of nodes that can effectively be handled is
    lower, typically 10..20 thousand. 
\end{verbatim}

\begin{verbatim}
Software:
- Lora libraries         : Semtech, with modifications from Beta Research BV
- basic packet forwarder : Semtech, 
- poly packet forwarder  : Beta Research BV, based on code from Semtech
- Installation scripts   : Beta Research BV
- Beagle Bone            : Various, see website beagle board.
\end{verbatim}

\section{Maintenance}
The Lorank does not need special maintenance. In case the device fails,
and this failure cannot be attributed to incorrect handling, please
contact Ideetron.
 
\section{Disposal}
This is an electronic device, and disposal should be in accordance with
local environmental regulations.

\section{Legal}
Ideetron can in no way be held responsible for
malfunctions and/or damage resulting from the
information presented in this document. 


\section{Questions and Answers.}

\begin{description}

\item[My gateway ID starts with 1DEE, why is that?] 
All gateways from Ideetron are preconfigured with an eight byte unique
ID starting with `1DEE' augmented with six randomly chosen bytes. This is 
to make the gateway easily recognisable in the (long) list of gateways at 
for example Semtech. This prefix is not registered anywhere, and you are 
completely free to change it to  your linking. If you keep the rest of 
the ID unchanged you can be reasonably sure the ID stays unique. 

\item[Are there any local tools to test my gateway?]
The gateway comes with some precompiled tools from Semtech, which
can be used for testing purposes. These can be found in the directories 
that match \url{/root/lorank8-[version_number]/lora_gateway/util_*}. Please see the
\url{readme.md} file therein for further instructions. Ideetron cannot provide
instructions for operation or guarantee for their quality of operation.

\item[How do I setup a server within my network to catch packets?]
If you want to receive packets on your own computer you need to set up
a router backend yourself. Or maybe you want to simulate the existence
of many node to test your setup. To that end download and compile the
lora\_simulator tools on the github of The Things Network:
\url{https://github.com/TheThingsNetwork/lora_simulator}.



\end{description}


\end{document} 










